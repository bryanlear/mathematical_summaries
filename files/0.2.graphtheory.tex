\documentclass[../Main.tex]{subfiles}

\begin{document}
\chapter{Combinatorics}

\section{Basic Methods}

\thmp{Pigeon-hole Principle}{Let $n$ and $k$ be positive integers, and let $n > k$. Suppose we have to place $n$ identical balls into $k$ identical boxes. Then there will be at least one box in which we place at least two balls.}{

Assume that statement $= FALSE =$ No box $\geq$ 2 balls. Therefore, $k$ boxes contain either 0 or 1 ball.

Let $m$ be the number of boxes that have zero balls.
Then $m \geq 0$.

Number boxes with 1 ball $= k-m$.
Total number balls placed in $k$ boxes is $1 \cdot (k-m) + 0 \cdot m = k-m$.

We are given that we placed $n$ balls into the boxes.
So, total number of balls is $n$.
Therefore, we must have $n = k-m$.

Since $m \geq 0$, it follows that $k-m \leq k$.
Thus, $n \leq k$.

This contradicts our initial assumption that $n > k$.
Therefore, our assumption that there is no box with at least two balls must have been false, and consequently, there is at least one box with at least two balls.
}

\textbf{\textit{'If you have more items ("pigeons") than you have containers ("pigeonholes"), then at least one container must hold more than one item.'}}\\

\exm{Pigeonhole Principle and modular arithmetic}{
For infinite sequence of numbers $a1=7, a_2=77, a_3=777,...,$ proof there is at least 1 number that is perfectly divisible by 2023. 
Proof even stronger statement: One of the first 2023 elements of the sequence  must be divisible by 2023.

\begin{enumerate}

    \item \textbf{Proof by Contradiction:} Assume \textbf{no element} in the sequence is divisible by 2023. Show assumption leads to logical impossibility.
    \item \textbf{Pigeons and pigeonholes:} First 2023 $=$ pigeons.
    Now, consider the \textbf{remainder} of numbers divided by 2023. Based on initial assumption, remainder \textbf{cannot be 0} and therefore only possible remainders are integers from 1 t0 2022. Thus exactly \textbf{2022 possible values} for remainder $=$ Pigeonholes.

    e.g.,
    
10 / 5 = 2 remainder 0\\
11 / 5 = 2 remainder 1\\
12 / 5 = 2 remainder 2\\
13 / 5 = 2 remainder 3\\
14 / 5 = 2 remainder 4\\
15 / 5 = 3 remainder
    
    Remainder can never be $\geq$ 5. Therefore, for any divisor $D$ remainder set $= \{0,1,2,...,2022\}$ ($2023$ elements).
    \item \textbf{Apply principle:} Pigeons $>$ pigeonhole. The principle guarantees $\geq$ 2 pigeons in 1 pigeonhole. Meaning, 2 elements $\geq$ must be placed in the same pigeon's hole $a_j$ and $a_i$, where $j > i$ must have \textbf{exact same remainder} when divided by 2023.\\

    $a_j = 2023 * k_j + r$\\
    
    $a_i = 2023 * k_i + r$ (where $k_j$ and $k_i$ are integers)\\

    $a_j - a_i = (2023 * k_j + r) - (2023 * k_i + r) = 2023(k_j-k_i)$, proving difference $a_i - a_j$ \textbf{must be perfectly divisible by 2023.}\\

    \textbf{e.g.}, given $j=5$ and $i=2$,
    $a_5 = 77777$\\
    $a_2 = 77$\\
    $a_5-a_2=77700 = a_3$\\
    $777*100 = a_3 * 10^2$\\
    
    $a_j-a_i=a_{j-i} * 10^i$
    \item{\textbf{Final contradiction:}} $a_{j-1}$ must be divisible by 2023. \textit{'If integer $N$ divides product $A*B$ and $N$ shares no common factors (\textbf{relatively prime)} with $B$, then $N$ must divide $A$}. Thus $2023$ and $10^i$ are relatively prime and $2023$ divides product $a_{j-i} * 10^i$ and prime to $10^i$, it \textbf{must divide} $a_{j-i}$.

    $a_{j-i}$ consists of $(j-i)$ sevens which means it is element of original sequence ($a1=7, a_2=77, a_3=777,...,$). THUS assumption MUST be $FALSE$ and original statement MUST be $TRUE$. 
\end{enumerate}

}
\thmp{General version, Pigeon-Hole}{
Let $n, m, r$ be positive integers so that $n >rm$, and let us distribute $n$ identical balls into $m$ identical boxes. There will be at least 1 box into which we place at least $r+1$ balls. }{
Assume contrary statement. Then each of the $m$ boxes can hold at most $r$ balls, so all boxes can hold at most $rm<n$ balls, which contradicts the requirement that we distribute $n$ balls.
}

\exm{Geometric application}{

\subsection*{$\geq 2/10$ within $0.48$}

\textbf{Given:}
\begin{itemize}
    \item Square of unit size (1*1)
    \item 10 points placed anywhere within square
\end{itemize}

\textbf{Prove:}
\begin{itemize}
    \item There must be $\geq$ 2/10 points that are closer to each other than a distance of $0.48$
    \item There must be $\geq$ 3/10 points that can be covered by a single disk of radius $0.5$
\end{itemize}

\textbf{Apply pigeon hole principle:}
\begin{itemize}
    \item \textbf{Pigeon holes:} First divide unit square into $9$, thus each box $1/3$ side length $\rightarrow$ pigeon holes.
$10$ total points \textbf{(pigeons)} into $9$ holes. \textit{At least $1$ hole must contain $>$ 1 pigeon}. Therefore, \textbf{at least $1/9$ squares must contain $2/10$ points.}
\end{itemize}

\textbf{Calculate max. distance within square(hole):}
\begin{itemize}
    \item Longest distance between any 2 points inside a square is length of diagonal.
    \item Since $a^2 + b^2 = c^2$ then diagonal$^2 = 2/9$. Diagonal is then $\sqrt{2} / 3$ $\approx 0.4714$. Diagonal $< 0.48$. 
    \item Maximum possible distance between 2 points in same square is $< 0.48$ thus 2 points \textit{must} exist in the square and these are closer to each other than $0.48$
\end{itemize}


\subsection*{$\geq 3/10$ covered by disk of radius $0.5$}

\textbf{Apply pigeonhole principle:}
\begin{itemize}
    \item Divide square into $4$ equal triangles using $2$ main diagonals (\textbf{pigeonholes}).
    \item If $N$ items are put into $k$ containers then at least $1$ container must hold at least $\lceil N/k \rceil$ items, where $\lceil \cdot \rceil$ denotes ceiling function (rounding up)

     $N=10$ (points) and $k=4$ (triangles). The calculation is:
\[
\left\lceil \frac{N}{k} \right\rceil = \left\lceil \frac{10}{4} \right\rceil = \lceil 2.5 \rceil = 3
\]
Thus at least $1/4$ triangles must contain \textit{at least $3$ points}.
\end{itemize}

\textbf{Geometric Argument - The Circumcircle}
The \textbf{circumcircle} of a triangle is the unique circle that passes through all three of its vertices. Key property is entire area of triangle is contained within its circumcircle. Therefore, proof that the circumcircle of each of $4$ triangles has radius $\leq 0.5$.
\begin{itemize}
    \item Imagine coordinates with set $V$ $=\{(0,0), (1,0), (1,1), (0,1)\}$. $2$ diagonals intersect at center of square $(0.5, 0.5)$.
    \item Consider e.g., triangle with vertexes at $(0,0), (0,1)$ and center $(0.5, 0.5)$. Circumcircle is thus circle that passes through said points.
    \item Center of circle can be found at $(0.5, 0)$. $R$ is thus 
    \[
        R = \text{distance}((0.5,0), (0,0)) = \sqrt{(0.5-0)^2 + (0-0)^2} = \sqrt{0.5^2} = 0.5
    \]
    \item By symmetry, all triangles formed by diagonals have circumcircle of $R=$ 0.5. It then follows that there must be $3$ points that can be covered by a disk of radius $0.5$.
\end{itemize}
}

\end{document}

